%
%	Packages
%
 \message{*[paquetes]* Packages (lots).................................[sub defs]}

%\RequirePackage{ifthen}
   \IfFileExists{ifthen.sty}
	{\usepackage{ifthen}
	 \typeout{... using Package ``ifthen.sty''}}
	{\typeout{Package ``ifthen.sty'' was not found.}}

 \RequirePackage{amsmath}
 \RequirePackage{amsfonts}
 \RequirePackage{amssymb}
 \usepackage{latexsym}
 
 \RequirePackage{pdfsync}
 \usepackage{pdfscreen}   % for Slides
 
 \RequirePackage{color}

%\RequirePackage{TikZdefinitions}
% - includes TikZbondGraph, TikZneuralNets by Benito R. Fernandez
 \typeout{... Trying to load --------> Package TikZdefinitions.sty ...}
   \IfFileExists{Misc/TikZdefinitions.sty}
	{\usepackage{Misc/TikZdefinitions}
	 \typeout{*[paquetes]* <--------- Package TikZdefinitions.sty (already) loaded.}
	 \setboolean{TikZloaded}{true}
	}
	{\typeout{Package TikZdefinitions.sty was not found.}
	 \typeout{*[paquetes]* <--------------- Package TikZdefinitions.sty not loaded.}
	 \setboolean{TikZloaded}{false}
	}
 \usepackage{makeidx}
 \usepackage[latin1]{inputenc}

%\RequirePackage{pstricks}
 \RequirePackage{snapshot}
 \usepackage{float}
 \usepackage{longtable}
% \usepackage{bm}
 \usepackage{type1cm} % allow arbitrary scaling
 \usepackage{relsize}
 \usepackage{url}
% \usepackage{bar}
% \usepackage{barddoc}
% \usepackage{subeqn}
 \usepackage{subeqnarray}
 \usepackage{caption}    % <- \usepackage{subfig} <- % \usepackage{subfigure}
 \usepackage{subcaption} % <- \usepackage{subfig} <- % \usepackage{subfigure}
 \usepackage{wrapfig}
 \usepackage{multirow}
% \usepackage{palatino}
 \usepackage{mflogo}		% for METAFONT 
% \usepackage{emp}		    %       <<<<<< WHAT IS THIS?
 \usepackage{pifont}		% for \ding{#} macros
 \usepackage{fancyhdr}
 \usepackage{smartdiagram}  % <-  for samrt diagrams

 \usepackage{xspace}
 
 \usepackage{listings}
 
 \lstdefinestyle{customMatLab}{
  title=\lstname,            % show the filename of files included with 
                             % \lstinputlisting; also try caption instead of title
  belowcaptionskip=1\baselineskip,
  language=Matlab,           % choose the language of the code
  basicstyle=\footnotesize\ttfamily, % the size of the fonts that are used  
  numbers=left,              % where to put the line-numbers
  xleftmargin=\parindent,    % left margin separation
  breaklines=true,           % sets automatic line breaking
  stepnumber=1,              % the step between two line-numbers. 
                             % If it's 1 each line will be numbered
  numbersep=2mm,             % how far the line-numbers are from the code
  backgroundcolor=\color{white}, % choose the background color.  
  							   % You must add \usepackage{color}
  showspaces=false,          % show spaces adding particular underscores
  showstringspaces=false,    % underline spaces within strings
  showtabs=false,            % show tabs within strings adding particular  
  							   % underscores
  frame=single,              % adds a frame around the code
  tabsize=2,                 % sets default tabsize to 2 spaces
  captionpos=b,              % sets the caption-position to bottom
  stringstyle=\color{orange},% string literal style
  commentstyle=\color{red}, % comment style
  keywordstyle=\bfseries\color{blue}, % keyword style
  identifierstyle=\color{marron},
  numberstyle=\tiny\color{gray}, % the style that is used for the line-numbers
  rulecolor=\color{morado},  % if not set, the frame-color may be changed on 
                             % line-breaks within not-black text
  breaklines=true,           % sets automatic line breaking
  breakatwhitespace=false,   % sets if automatic breaks should only happen  
                             % at whitespace
  escapeinside={\%*}{*)}     % if you want to add a comment within your code
  extendedchar=true,         % lets you use non-ASCII characters; for 8-bits  
 }%\lstdefinestyle
 
 \newcommand{\includecode}[2][MatLab]{\lstinputlisting[caption=#2, 
                                 escapechar=@, style=custom#1]{#2}}
 
 \typeout{*[paquetes]* <--------------- Packages loaded!}
