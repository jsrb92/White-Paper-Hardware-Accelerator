%
%    Macros for Partial Derivatives and other Math Operators
%
 \message{*[ProgrammingMacros]* Programming macros.............................[sub defs]}
%
\message{Listings with ProcNames,}
%
 \usepackage[procnames]{listings}
 \usepackage{color}
 \usepackage{caption}
%
% Default fixed font does not support bold face
\DeclareFixedFont{\ttb}{T1}{txtt}{bx}{n}{12} % for bold
\DeclareFixedFont{\ttm}{T1}{txtt}{m}{n}{12}  % for normal

% 
 \def\Python{{\color{azulOscuro}\cool{Python}}\xspace}
 \def\SciPy{{\color{rojoOscuro}\cool{SciPy}}\xspace}
 \def\IPython{{\color{cyan}\cool{IPython}}\xspace}
 \def\Matplotlib{{\color{morado}\cool{Matplotlib}}\xspace}
 \def\NumPy{{\color{verdeOscuro}\cool{NumPy}}\xspace}

%--------------------------------------------------------
%--------------------------------------------------------
% Programming style for highlighting code: Python
%--------------------------------------------------------
%--------------------------------------------------------
% Custom colors
 \definecolor{keywords-color}{RGB}{0,0,113}       % dark blue
 \definecolor{comments-color}{RGB}{255,0,90}      % red
 \definecolor{strings-color}{RGB}{0,150,0}        % green
 \definecolor{identifiers-color}{RGB}{160,0,0}    % dark red
 \definecolor{emphasis-color}{rgb}{0.6,0,0}       % deep red
 \definecolor{codeTitle-color}{rgb}{0.6,0,0}      % burnt orange: burntOrange!90
 \definecolor{background-color}{rgb}{1,1,1}       % white
 \definecolor{codeBox-color}{rgb}{0.95,0.95,0.95} % very light gray
% Custom styles for boxes & title
\tikzstyle{sourceCodeBox} = [ draw=blue, fill=codeBox-color, ultra thick,
                              rectangle, rounded corners, inner sep=5mm ]

\tikzstyle{sourceCodeTitle} = [ fill=burntOrange!90, text=white, very thick,
                                rectangle, rounded corners ]
%
\message{Box Environment,}
%
% font styles: \normalfont\em\rmfamily\sffamily\\ttfamily\upshape\itshape\slshape
%              \scshape\bfseries\mdseries\lfseries
\newcommand\pythonStyle[1][\scriptsize]{\lstset{%
        language=Python,
% font sizes: \tiny\scriptsize\footnotesize\small\normalsize\large\Large\LARGE\huge\Huge
  		backgroundcolor=\color{background-color},  % choose the background color
        basicstyle=\ttfamily#1, 			       % font family and size
    	otherkeywords={self},                      % Add keywords here
        keywordstyle=\color{keywords-color},       % keywords color	
        commentstyle=\color{comments-color},       % comments color
        stringstyle=\color{strings-color},         % strings color
        identifierstyle=\color{identifiers-color}, % identifiers color
        procnamekeys={def,class}
    	emph={MyClass,__init__},             % Custom highlighting
    	emphstyle=\color{emphasis-color},    	     % Custom highlighting style
        escapeinside={(*@}{@*)},
        escapeinside={\%*}{*)},              % if you want to add LaTeX within your code
        %numbers=left,
        breaklines=true,
        breakatwhitespace=true,
        showspaces=false,
        showstringspaces=false,
        frame=shadowbox,
        fill=white,
        frameround=rrrt,
        linewidth=.80\linewidth,
        rulecolor=\color{magenta},
        rulesepcolor=\color{morado!20}
    }}%

% Define Caption for code listings
\DeclareCaptionFont{white}{ \color{white} }
\DeclareCaptionFormat{listing}{
  \colorbox[cmyk]{0.43, 0.35, 0.35,0.01 }{
    \noindent\parbox{.80\linewidth}{\hspace{15pt}#1#2#3}
  }
}

\captionsetup[lstlisting]{ format=listing, labelfont=white, textfont=white, singlelinecheck=false, margin=0mm, font={bf,footnotesize} }


%
\message{Python,}
%
% Python for inline
\newcommand\pythonInLine[1]{{\pythonStyle[\small]\lstinline!#1!}}

% Python for external files
\newcommand\pythonExternal[2][]{{
\pythonStyle
\lstinputlisting[#1]{#2}}}

% Python environment
\lstnewenvironment{python}[1][]
{
\pythonStyle
\lstset{#2}
}
{}

%%%%%%%%%%%%%%%%%%%%%%%%%%%%%%%%%% Do similar to multiple pages tables! For long code!!!!
%\newcommand{pythonCode}[1]{
\makeatletter
\lstnewenvironment{pythonCode}[1][]{%
    \def\pythonCodeTitle{#1}%
    \pythonStyle%
    \setbox\@tempboxa=\hbox\bgroup\color@setgroup
}%
{%
    \color@endgroup\egroup
    \begin{tikzpicture}
        \node[sourceCodeBox] (box)
            % Makebox is needed to take the frame added by listings into account
            {\makebox[.90\linewidth][l]{\box\@tempboxa}};
        \node[sourceCodeTitle] at (box.north west) {\pythonCodeTitle};
%        \node[sourceCodeTitle] at ($(box.north west)+(5mm,0)$) {\pythonCodeTitle};
    \end{tikzpicture}
}
%--------------------------------------------------------
%--------------------------------------------------------
%--------------------------------------------------------
%
\message{Swift,}
%

%
\message{MatLab,}
%


%
\message{Stickies.}
%

 \newcommand\stickie[1]{
   \tikz\node[overlay,fill=yellow!50,draw,thick, 
                      minimum height=2cm, minimum width=3cm,rotate=30,
                      decorate, decoration={random steps,
                      segment length=3pt,amplitude=1pt}]
                   at (0.9\textwidth,20mm){#1};
 }

