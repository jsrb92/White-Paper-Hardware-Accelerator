%
%     Picture Macros
%
 \message{*[pictureMacros]* Pictures macros.................................[sub defs]}

 \def\picture #1 by #2 (#3){
  \vbox to #2{
     \hrule width #1 height 0pt depth 0pt
     \vfill
     \special{picture #3}}
}
%
%    Scaled  Picture Macro
%
%\def\scaledpicture #1 by #2 (#3 scaled #4){
%  {
%   \dimen0=#1 \dimen1=#2
%   \multiply\dimen0 by #4 \divide\dimen0 by 1000 
%   \multiply\dimen1 by #4 \divide\dimen1 by 1000 
%   \picture \dimen0 by \dimen1 (#3)
%  }
%}
\def\scaledpicture #1 by #2 (#3 scaled #4){
   \ancho=#1 \alto=#2 \escala=#4
   \multiply\ancho by \escala \divide\ancho by 1000  
   \multiply\alto  by \escala \divide\alto  by 1000 
%   File:``#3'', scaled is \number\ancho\ by \number\alto\ \\
   \includegraphics[width=\number\ancho in, height=\number\alto in]{#3}}
   
 \newcount\ancho  \newcount\alto   \newcount\escala 
 \def\myScaledpicture #1 by #2 (#3 scaled #4){
  {
   \dimen0=#1 \dimen1=#2
%   File:``#3'', scaled is \number\dimen0 \ by \number\dimen1 \ \\
   \multiply\dimen0 by #4 \divide\dimen0 by 1000 
   \multiply\dimen1 by #4 \divide\dimen1 by 1000
%   File:``#3'', scaled is \number\dimen0 \ by \number\dimen1 \ \\
   \includegraphics[width=\dimen0 in, height=\dimen1 in]{#3}
  }
}
% \newcommand{\myScaledpicture}[4]{
%  {
%   \newcount\ancho \newcount\alto 
%   \ancho=#1 \alto=#2
%   \multiply\ancho by #4 \divide\ancho by 1000 
%   \multiply\alto  by #4 \divide\alto  by 1000 
%   \includegraphics[widht=\number\ancho][height=\number\alto]{#3}
%  }
%}

%
%     Graphics Macros
%
 \message{Graphics,}
 \usepackage{Misc/bez123}

%
\def\linea{\vskip\baselineskip      % horizontal line
           \hrule                   % horizontal line
           \vskip\baselineskip}     % horizontal line
 \def\nsfline{\hrule
%  \begin{picture}(440,1)(0,0)
%  \put(0,0){\line(1,0){400}}
%  \end{picture} 
%  \vspace*{0.2in}
 }
 
\def\myCircle(#1,#2)[#3]{ % (#1=xCenter, #2=yCenter) [#3=scale-radius]
	\put(#1,#2){
		\setlength{\unitlength}{#3}
		\cbezier(-5, 0)(-5, 2.761423749)(-2.761423749, 5)( 0, 5) % NW quadrant
		\cbezier( 0, 5)( 2.761423749, 5)( 5, 2.761423749)( 5, 0) % NE quadrant
		\cbezier( 5, 0)( 5,-2.761423749)( 2.761423749,-5)( 0,-5) % SE quadrant
		\cbezier( 0,-5)(-2.761423749,-5)(-5,-2.761423749)(-5, 0) % SW quadrant
		}
	}

\def\myOval(#1,#2){ % #1=radius, #2=linethinckness
	  	\linethickness{#2}
		\qbezier( #1,   0)( #1,  #1)(  0,  #1)
	  	\qbezier(  0,  #1)(-#1,  #1)(-#1,   0)
	  	\qbezier(-#1,   0)(-#1, -#1)(  0, -#1)
	  	\qbezier(  0, -#1)( #1, -#1)( #1,   0)
}

 \def\sqbullet{\vrule height .9ex width .8ex depth -.1ex }        % square bullet

